\section{Results}
\label{sec:results}

\para{Main comparison.} \Tabref{tab:main_results} reports the primary 5 bps results. Among LLM variants, \posaware is best on risk-adjusted return (Sortino 2.8352, Sharpe 2.0966), while \monthlyllm has the lowest drawdown (-0.1347) and lower turnover than weekly. Daily LLM is consistently weaker and has the highest turnover (59.4933).

\begin{table*}[t]
\centering
\resizebox{\textwidth}{!}{%
\begin{tabular}{@{}lrrrrrr@{}}
\toprule
Method & CumReturn & AnnRet & Sharpe & Sortino & MaxDD & Turnover \\
\midrule
\momwk & {\bf 0.4527} & {\bf 0.4169} & {\bf 2.2372} & {\bf 2.9801} & -0.1613 & 16.9333 \\
\posaware & 0.4261 & 0.3928 & 2.0966 & 2.8352 & -0.1518 & 24.2000 \\
\memoryless & 0.4274 & 0.3940 & 2.0820 & 2.7305 & -0.1660 & 37.2333 \\
\monthlyllm & 0.3804 & 0.3511 & 1.8274 & 2.4603 & {\bf -0.1347} & 12.0333 \\
\dailyllm & 0.3091 & 0.2858 & 1.6095 & 2.1650 & -0.1432 & 59.4933 \\
\ew & 0.2841 & 0.2629 & 1.4577 & 1.8574 & -0.1897 & {\bf 0.0000} \\
\invvolwk & 0.2260 & 0.2094 & 1.3125 & 1.7189 & -0.1752 & 6.3687 \\
\bottomrule
\end{tabular}%
}
\caption{Main results on 2025-01-02 to 2026-01-30 with 5 bps transaction costs. Higher is better except MaxDD where values closer to zero are better. Best values are in bold.}
\label{tab:main_results}
\end{table*}


\para{Cadence effect within LLMs.} Weekly vs daily improves Sortino by +0.6702 and cuts turnover by 35.2933. Monthly vs daily improves Sortino by +0.2953 and cuts turnover by 47.4600. These results support the claim that slower control reduces churn and improves practical risk-adjusted behavior.

\para{Statistical tests.} Weekly position-aware versus daily position-aware yields a significant Wilcoxon signed-rank result ($p=0.0272$) on paired daily returns, with mean daily difference $+3.18\times10^{-4}$. The bootstrap 95\% CI for Sortino difference is wide ([-0.355, 2.073]), indicating directional improvement with moderate uncertainty. Monthly versus daily is not significant ($p=0.4382$). Weekly position-aware versus weekly memoryless is also not significant on daily returns ($p=0.5787$), suggesting the main gain comes from cadence rather than per-day return distribution shifts.

\para{Ablation on position-awareness.} \posaware and \memoryless reach similar cumulative return (0.4261 vs 0.4274), but position-awareness lowers drawdown (-0.1518 vs -0.1660) and turnover (24.20 vs 37.23). This supports position continuity as a stability mechanism.

\para{Comparison to non-LLM baselines.} \momwk remains best overall (Sortino 2.9801, Sharpe 2.2372), outperforming all LLM variants in this window. Still, weekly and monthly LLM beat \ew and \invvolwk on major risk-adjusted metrics.

\begin{figure}[t]
\centering
\includegraphics[width=0.95\linewidth]{figures/equity_curves.png}
\caption{Equity curves for all methods on the 2025-01-02 to 2026-01-30 test period. Weekly and monthly LLM policies stay above daily LLM for most of the window, while weekly momentum remains highest overall.}
\label{fig:equity}
\end{figure}

\begin{figure}[t]
\centering
\includegraphics[width=0.95\linewidth]{figures/drawdown_curves.png}
\caption{Drawdown trajectories. Monthly LLM has the shallowest trough among LLM variants, and weekly position-aware avoids some deeper drops seen in weekly memoryless.}
\label{fig:drawdown}
\end{figure}

\begin{figure}[t]
\centering
\includegraphics[width=0.95\linewidth]{figures/return_boxplot.png}
\caption{Distribution of daily returns. Distribution overlap explains small effect-size estimates even when cumulative and turnover outcomes diverge materially.}
\label{fig:boxplot}
\end{figure}

\para{Transaction-cost robustness.} Under 0, 5, and 10 bps settings, weekly and monthly LLM policies remain above daily LLM in key risk-adjusted metrics. This pattern suggests the cadence advantage is not an artifact of one cost assumption.
